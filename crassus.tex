\title{The Crassus Token}
\author{
        Joey Lemberg \\
        Paul Gibler\\}
\date{\today}

\documentclass[12pt]{article}

\begin{document}
\maketitle

\begin{abstract}
The Crassus token is a cryptocurrency traded on the Ethereum blockchain.  It is designed to maximize accessibility and ease of use.  Crassus leverages the ERC20 token infrastructure among features of the Ethereal platform to deliver a reliable cryptocurrency with many highly desired attributes.
\end{abstract}

\section{Introduction}
The most popular use for blockchain technology is cryptocurrency.  The most performant and supported blockchain is provided by Ethereum.  Crassus attempts to provide the most usable cryptocurrency on the Ethereum blockchain.

\paragraph{Outline}
The remainder of this paper explains the design decisions which make Crassus an attractive cryptocurrency.

\section{Ethereum and ERC20}\label{Ethereum and ERC20}
The advantages of using the Ethereum platform are clear.  By using the Ethereum platform, Crassus is bestowed with the security, trust, reliability, and speed of Ethereum.  Furthermore, by implementing the ERC20 token standard, Crassus comes with a complete toolset for users of Crassus, allowing users to view balances and trade Crassus using their current Ethereum wallet and preferred Ethereum software.

The Ethereum platform and ERC20 token standard clear provide a strong basis for Crassus.

\section{Supply Details}\label{Supply Details}
There are 900,000 total Crassus, with the smallest unit of Crassus being 0.0001.  These values were selected with the goal of usability.

The 4 decimal limit makes values of Crassus easy to read.  Unlike with other cryptocurrencies, users can infer values of Crassus without needing to count decimals or use conversion software.  This also gives users a reason to value all quantities of Crassus.  In this way, Crassus solves the "penny problem," where the smallest unit a currency is perceived as worthless.

The low total supply was chosen to help bolster the value of the Crassus.  The psychological principal that rarity correlates to perceived value appears to hold true with cryptocurrencies.  A total supply of less than one million, combined with the limited decimal places, makes Crassus a very rare cryptocurrency, and strengthens the token's ability to hold value.


\section{Branding}\label{Branding}
Branding of a cryptocurrency is supremely important in determining success.  Bitcoin Cash is most distinguished by it's close resemblance to bitcoin in name and image.  Slogans such as "digital gold" have been instrumental in convincing the world that Bitcoin is truly valuable.

Crassus' name and logo are designed to inspire memories of Ancient Rome, a proud high water mark in the history of our civilization.  Most people in cryptocurrency today feel the gears of history turning, and as the world changes around us, there is comfort to be found, and lessons to be learned from looking back in history.  Crassus helps to ground its users in the many millennia long history of money and democracy in which we are all living.

\bibliographystyle{abbrv}
\bibliography{main}

\end{document}  